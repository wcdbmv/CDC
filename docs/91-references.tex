%% ГОСТ 7.32-2017
%% 5.10 Список использованных источников
%%
%% 5.10.1 Список должен содержать сведения об источниках, использованных при составлении отчета.
%% Сведения об источниках приводятся в соответствии с требованиями ГОСТ 7.1, ГОСТ 7.80, ГОСТ 7.82.
%%
%% 5.10.2 Список использованных источников должен включать библиографические записи на документы, использованные при составлении отчета, ссылки на которые оформляют арабскими цифрами в квадратных скобках.
%% Список использованных источников оформляют в соответствии с 6.16.


\StructuralElement{Список использованных источников}

% да-да, очень плохо, похуй
\begin{enumerate}
	\item Ахо А., Сети Р., Ульман Д. Компиляторы: принципы, технологии и инструментарий, 2-е изд.: Пер. с англ. — М.: ООО «И.Д. Вильямс», 2008. — 1184 c.
	\item Серебряков В. А., Галочкин М. П. Основы конструирования
	компиляторов.
	\item Ахо А., Ульман Дж. Теория синтаксического анализа, перевода и компиляции. Том 1.: Пер. с англ. — М.: «Мир», 1978.
	\item Terence Parr. The Definitive ANTLR4 Reference. — М.: Pragmatic Bookshelf, 2013.
	\item Chris Lattner. The Design of LLVM. Dr. Dobb’s Journal, 2012.
\end{enumerate}
