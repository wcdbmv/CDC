%% Методические указания к выполнению, оформлению и защите выпускной квалификационной работы бакалавра
%% 2.6 Технологический раздел
%%
%% Технологический раздел содержит обоснованный выбор средств программной реализации, описание основных (нетривиальных) моментов разработки и методики тестирования созданного программного обеспечения.
%%
%% В этом же разделе описывается информация, необходимая для сборки и запуска разработанного программного обеспечения, форматы входных, выходных и конфигурационных файлов (если такие имеются), а также интерфейс пользователя и руководство пользователя.
%%
%% Если для правильного функционирования разработанного программного обеспечения требуется некоторая инфраструктура (веб-приложение, база данных, серверное приложение), уместно представить её с помощью диаграммы развёртывания UML.
%%
%% Как уже говорилось, часть технологического раздела должна быть посвящена тестированию разработанного программного обеспечения.
%%
%% Модульное тестирование описывается в технологическом разделе.
%%
%% Системное тестирование может быть описано в технологическом или экспериментальном разделах, в зависимости от глубины его реализации и тематики бакалаврской работы.
%%
%% При проведении тестирования разработанного программного обеспечения следует широко использовать специализированные программные приложения: различные статические анализаторы кода (например, clang); для тестирования утечек памяти в языках программирования, где отсутствует автоматическая «сборка мусора», Valgrind, Doctor Memory и их аналоги, и т. п.
%%
%% Рекомендуемый объём технологического раздела 20—25 страниц.


\chapter{Технологический раздел}

\section{Сгенерированные классы анализаторов}

В результате работы ANTLR будут сгенерированы следующие классы:
\begin{itemize}
	\item BasicLexer — лексический анализатор.
	\item BasicParser — синтаксический анализатор.
	\item BasicParserVisitor — абстрактный класс посетителя для обхода дерева.
	\item BasicParserListener — абстрактный класс слушателя, с пустыми методами.
\end{itemize}

Для обхода дерева, необходимо создать класс-наследник базового класса слушателя и переопределить enter- и exit-методы, соответствующие правилам исходной грамматики.

\section{Обнаружение ошибок}

Все ошибки, обнаруженные на этапах лексического и синтаксического анализов, обрабатываются средствами ANTLR.
Возникающие при этом исключения обрабатываются и выводится текст соответствующей ошибки.
Также при построении графа вызовов обрабатываются ошибки вызова функции до ее объявления.

\section{Пример компиляции программы}

Пример программы, написанный на рассматриваемом подмножестве языка BASIC представлен на листинге \ref{lst:bubble-bas}.
Данная программа реализует сортировку пузырьком.

\begin{lstlisting}[
	caption={Программа на рассматриваемом подмножестве языка BASIC},
	label={lst:bubble-bas},
	language={[Visual]Basic},
	gobble=8
]
	SUB Main
		DIM A(100)
		INPUT "Array size: ", n
		PRINT "Input array"
		FOR i = 1 TO n + 1
			INPUT A(i)
		END FOR

		FOR i = 0 TO n
			FOR j = 2 TO n - i + 1
				IF A(j) < A(j - 1) THEN
					LET tmp = A(j)
					LET A(j) = A(j - 1)
					LET A(j - 1) = tmp
				END IF
			END FOR
		END FOR

		FOR i = 1 TO n + 1
			PRINT A(i)
		END FOR
	END SUB
\end{lstlisting}


В листинге \ref{lst:bubble-ll} в приложении А приведен сгенерированный LLVM IR код для представленной на листинге \ref{lst:bubble-bas} программы.


%\section{Выводы}
