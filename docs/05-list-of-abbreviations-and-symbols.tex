%% ГОСТ 7.32-2017
%% 5.6 Перечень сокращений и обозначений
%%
%% 5.6.1 Структурный элемент "ПЕРЕЧЕНЬ СОКРАЩЕНИЙ И ОБОЗНАЧЕНИЙ" начинают со слов: "В настоящем отчете о НИР применяют следующие сокращения и обозначения".
%%
%% 5.6.2 Если в отчете используют более трех условных обозначений, требующих пояснения (включая специальные сокращения слов и словосочетаний, обозначения единиц физических величин и другие специальные символы), составляется их перечень, в котором для каждого обозначения приводят необходимые сведения.
%%
%% Допускается определения, обозначения и сокращения приводить в одном структурном элементе "ОПРЕДЕЛЕНИЯ, ОБОЗНАЧЕНИЯ И СОКРАЩЕНИЯ".
%%
%% 5.6.3 Если условных обозначений в отчете приведено менее трех, отдельный перечень не составляют, а необходимые сведения указывают в тексте отчета или в подстрочном примечании при первом упоминании.
%%
%% 5.6.4 Перечень сокращений и обозначений следует оформлять в соответствии с 6.15.


\StructuralElement{Перечень сокращений и обозначений}
