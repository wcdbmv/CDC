%% ГОСТ 7.32-2017
%% 5.11 Приложения
%%
%% 5.11.1 В приложения рекомендуется включать материалы, дополняющие текст отчета, связанные с выполненной НИР, если они не могут быть включены в основную часть.
%%
%% В приложения могут быть включены:
%% - дополнительные материалы к отчету;
%% - промежуточные математические доказательства и расчеты;
%% - таблицы вспомогательных цифровых данных;
%% - протоколы испытаний;
%% - заключение метрологической экспертизы;
%% - инструкции, методики, описания алгоритмов и программ, разработанных в процессе выполнения НИР;
%% - иллюстрации вспомогательного характера;
%% - копии технического задания на НИР, программы работ или другие исходные документы для выполнения НИР;
%% - протокол рассмотрения результатов выполненной НИР на научно-техническом совете;
%% - акты внедрения результатов НИР или их копии;
%% - копии охранных документов.
%%
%% 5.11.2 Приложения к отчету о НИР, в составе которых предусмотрено проведение патентных исследований, могут быть включены в отчет о патентных исследованиях, оформленный по ГОСТ 15.011, библиографический список публикаций и патентных документов, полученных в результате выполнения НИР, который должен быть оформлен по ГОСТ 7.1, ГОСТ 7.80, ГОСТ 7.82.
%%
%% 5.11.3 Приложения оформляются в соответствии с 6.17.

%% Методические указания к выполнению, оформлению и защите выпускной квалификационной работы бакалавра
%% 2.12 Приложения
%%
%% Приложения состоят из вспомогательного материала, на который в основной части бакалаврской работы имеются ссылки.
%% Приложением оформляют различные схемы, листинг программ, наборы тестов и др.
%%
%% В тексте РПЗ на все приложения должны быть даны ссылки.


\StructuralElement{Приложение А}

\begin{lstlisting}[
	basicstyle=\ttfamily\scriptsize,
	caption={Сгенерированный LLVM IR код},
	label={lst:bubble-ll},
	language={LLVM},
	gobble=8,
	]
	@g_str = private unnamed_addr constant [13 x i8] c"Array size: \00", align 1
	@g_str.1 = private unnamed_addr constant [12 x i8] c"Input array\00", align 1
	@g_str.2 = private unnamed_addr constant [2 x i8] c"?\00", align 1
	@.str = private unnamed_addr constant [4 x i8] c"%s \00", align 1
	@.str.1 = private unnamed_addr constant [4 x i8] c"%lf\00", align 1
	@.str.2 = private unnamed_addr constant [5 x i8] c"%lf\0A\00", align 1
	@stdin = external global %struct._IO_FILE*, align 8
	@.str.3 = private unnamed_addr constant [4 x i8] c"%s\0A\00", align 1

	define void @Main() {
		label_start:
		%A_addr = alloca double, i32 100, align 8
		%n_addr = alloca double, align 8
		%i_addr = alloca double, align 8
		%j_addr = alloca double, align 8
		%tmp_addr = alloca double, align 8
		%0 = call double @bsq_number_input(i8* getelementptr inbounds ([13 x i8], [13 x i8]* @g_str, i32 0, i32 0))
		store double %0, double* %n_addr, align 8
		call void @bsq_text_print(i8* getelementptr inbounds ([12 x i8], [12 x i8]* @g_str.1, i32 0, i32 0))
		store double 1.000000e+00, double* %i_addr, align 8
		%n = load double, double* %n_addr, align 8
		%add = fadd double %n, 1.000000e+00
		br label %1

		1:                                                ; preds = %4, %label_start
		%2 = load double, double* %i_addr, align 8
		%3 = fcmp olt double %2, %add
		br i1 %3, label %4, label %11

		4:                                                ; preds = %1
		%5 = call double @bsq_number_input(i8* getelementptr inbounds ([2 x i8], [2 x i8]* @g_str.2, i32 0, i32 0))
		%i = load double, double* %i_addr, align 8
		%6 = fptosi double %i to i32
		%7 = add i32 -1, %6
		%8 = getelementptr double, double* %A_addr, i32 %7
		store double %5, double* %8, align 8
		%9 = load double, double* %i_addr, align 8
		%10 = fadd double %9, 1.000000e+00
		store double %10, double* %i_addr, align 8
		br label %1

		11:                                               ; preds = %1
		store double 0.000000e+00, double* %i_addr, align 8
		%n1 = load double, double* %n_addr, align 8
		br label %12

		12:                                               ; preds = %48, %11
		%13 = load double, double* %i_addr, align 8
		%14 = fcmp olt double %13, %n1
		br i1 %14, label %15, label %51

		15:                                               ; preds = %12
		store double 2.000000e+00, double* %j_addr, align 8
		%n2 = load double, double* %n_addr, align 8
		%i3 = load double, double* %i_addr, align 8
		%sub = fsub double %n2, %i3
		%add4 = fadd double %sub, 1.000000e+00
		br label %16

		16:                                               ; preds = %45, %15
		%17 = load double, double* %j_addr, align 8
		%18 = fcmp olt double %17, %add4
		br i1 %18, label %19, label %48

		19:                                               ; preds = %16
		br label %20

		20:                                               ; preds = %19
		%j = load double, double* %j_addr, align 8
		%21 = fptosi double %j to i32
		%22 = add i32 -1, %21
		%23 = getelementptr double, double* %A_addr, i32 %22
		%24 = load double, double* %23, align 8
		%j5 = load double, double* %j_addr, align 8
		%sub6 = fsub double %j5, 1.000000e+00
		%25 = fptosi double %sub6 to i32
		%26 = add i32 -1, %25
		%27 = getelementptr double, double* %A_addr, i32 %26
		%28 = load double, double* %27, align 8
		%lt = fcmp olt double %24, %28
		br i1 %lt, label %29, label %44

		29:                                               ; preds = %20
		%j7 = load double, double* %j_addr, align 8
		%30 = fptosi double %j7 to i32
		%31 = add i32 -1, %30
		%32 = getelementptr double, double* %A_addr, i32 %31
		%33 = load double, double* %32, align 8
		store double %33, double* %tmp_addr, align 8
		%j8 = load double, double* %j_addr, align 8
		%sub9 = fsub double %j8, 1.000000e+00
		%34 = fptosi double %sub9 to i32
		%35 = add i32 -1, %34
		%36 = getelementptr double, double* %A_addr, i32 %35
		%37 = load double, double* %36, align 8
		%j10 = load double, double* %j_addr, align 8
		%38 = fptosi double %j10 to i32
		%39 = add i32 -1, %38
		%40 = getelementptr double, double* %A_addr, i32 %39
		store double %37, double* %40, align 8
		%tmp = load double, double* %tmp_addr, align 8
		%j11 = load double, double* %j_addr, align 8
		%sub12 = fsub double %j11, 1.000000e+00
		%41 = fptosi double %sub12 to i32
		%42 = add i32 -1, %41
		%43 = getelementptr double, double* %A_addr, i32 %42
		store double %tmp, double* %43, align 8
		br label %45

		44:                                               ; preds = %20
		br label %45

		45:                                               ; preds = %44, %29
		%46 = load double, double* %j_addr, align 8
		%47 = fadd double %46, 1.000000e+00
		store double %47, double* %j_addr, align 8
		br label %16

		48:                                               ; preds = %16
		%49 = load double, double* %i_addr, align 8
		%50 = fadd double %49, 1.000000e+00
		store double %50, double* %i_addr, align 8
		br label %12

		51:                                               ; preds = %12
		store double 1.000000e+00, double* %i_addr, align 8
		%n13 = load double, double* %n_addr, align 8
		%add14 = fadd double %n13, 1.000000e+00
		br label %52

		52:                                               ; preds = %55, %51
		%53 = load double, double* %i_addr, align 8
		%54 = fcmp olt double %53, %add14
		br i1 %54, label %55, label %62

		55:                                               ; preds = %52
		%i15 = load double, double* %i_addr, align 8
		%56 = fptosi double %i15 to i32
		%57 = add i32 -1, %56
		%58 = getelementptr double, double* %A_addr, i32 %57
		%59 = load double, double* %58, align 8
		call void @bsq_number_print(double %59)
		%60 = load double, double* %i_addr, align 8
		%61 = fadd double %60, 1.000000e+00
		store double %61, double* %i_addr, align 8
		br label %52
		
		62:                                               ; preds = %52
		ret void
	}

	define i32 @main() {
		start:
		call void @Main()
		ret i32 0
	}
\end{lstlisting}
