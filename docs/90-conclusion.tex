
%% ГОСТ 7.32-2017
%% 5.9 Заключение
%%
%% Заключение должно содержать:
%% - краткие выводы по результатам выполненной НИР или отдельных ее этапов;
%% - оценку полноты решений поставленных задач;
%% - разработку рекомендаций и исходных данных по конкретному использованию результатов НИР;
%% - результаты оценки технико-экономической эффективности внедрения;
%% - результаты оценки научно-технического уровня выполненной НИР в сравнении с лучшими достижениями в этой области.

%% Методические указания к выполнению, оформлению и защите выпускной квалификационной работы бакалавра
%% 2.10 Заключение
%%
%% Заключение содержит краткие выводы по всей работе и оценку полноты решения поставленной задачи.


\StructuralElement{Заключение}

В ходе работы над данным проектом был проведён анализ предметной области, разработан блок лексического и синтаксического анализа с явным построением дерева разбора для заданного исходного кода, разработан блок семантического анализа и генерации кода LLVM IR.

В результате чего был реализован компилятор подмножества языка BASIC с использованием ANTLR и LLVM.

